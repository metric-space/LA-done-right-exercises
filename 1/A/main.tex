%%% Local Variables:
%%% TeX-engine: default
%%% End:


\documentclass{article}
\usepackage[utf8]{inputenc}
\usepackage{amsmath}
\usepackage{amssymb}


\begin{document}

\subsubsection*{Q1}
\begin{equation*}
\begin{split}
    \frac{1}{a + b i} &= \frac{1}{a+ bi}* \frac{a - bi}{a - bi} \\
    &= \frac{a - bi}{a^{2} + abi - abi +b^2} \\
    &= \frac{a-bi}{a^2+b^2} \\
  c + di  &= \frac{a}{a^2+b^2} - \frac{bi}{a^2 + b^2}
\end{split}    
\end{equation*}
therefore $c = \frac{a}{a^2+b^2}$ and $d = \frac{-b}{a^2+b^2}$ \\


\subsubsection*{Q2}
\begin{equation*}
    \left( \frac{-1 + \sqrt{3}i}{2} \right)^3 = \frac{1}{8}*\left(-1 + \sqrt{3}i \right)^3
\end{equation*}
using the identity $\left(a + b \right)^3 = a^3 +3a^2b + 3ab^2 + b^3$ 
where $a = -1$ and $b = \sqrt(3)i$
we have
\begin{equation*}
    \begin{split}
        \frac{1}{8} * \left(-1 + \sqrt{3}i \right)^3 &= (-1)^3 + 3*(-1)^2*(\sqrt{3}i) + 3*(-1)*(\sqrt{3}i)^2 + (\sqrt{3}i)^3 \\
        &= \frac{1}{8} * \left( -1 + 3*\sqrt{3}i + 9 -3*\sqrt{3}i \right)  \\
        &= \frac{8}{8} \\
        &= 1
    \end{split}
\end{equation*}

\subsubsection*{Q3}
$-\sqrt{i}$ and $+\sqrt{i}$ are two distinct square roots of $i$ \\ 

\subsubsection*{Q4}
To show: 

\begin{equation*}
    \alpha + \beta = \beta + \alpha \textrm{ }\forall \alpha \textrm{,}\beta \in \mathbb{C}
\end{equation*}
let $\alpha = a +bi$ and $beta = c + di$ \\
by definition of addition of complex numbers \\
\begin{equation}
(a + bi) + ( c + di) = (a+c) + ( b + d)i
\end{equation}
by definition of complex numbers we know a,b,c,d $\in \mathbb{R}$ \\
so addition is commutative in $\mathbb{R}$ \\
as a consequence $ a + b = b + a$ and $ c + d = d + c $ \\
there $(1)$ becomes
\begin{equation*}
\begin{split}
    (a + bi) + (c + di) &= (a + c) + (b + d)i \\
     &= (c + a) + (d + b)i \\
     &= (c + di) + (a +  bi)  \textrm{ (from bidirectionality of how complex number addition is defined) } \\
     &= \beta + \alpha 
\end{split}
\end{equation*}
$\blacksquare$

\subsubsection*{Q5}
To prove:
\begin{equation*}
    \alpha + \left( \beta + \gamma \right) = \left( \alpha + \beta \right) + \gamma \textrm{ }\forall \alpha \textrm{,} \beta \textrm{,} \gamma \in \mathbb{C}
\end{equation*}
Assuming $\alpha = a + bi$, $\beta = c + di$, and $\gamma = e + fi$ \\
evaluating $\alpha + \left( \beta + \gamma \right)$
\begin{equation*}
    \begin{split}
        \alpha + \left( \beta + \gamma \right) &= (a + bi) + \Big( (c + di ) + (e + fi) \Big) \\
                                               &= (a + bi) + \Big( (c + e) + (d + f)i \Big)  \textrm{ (From addition of complex numbers)}\\
                                               &=  \Big(a + ( c + e) \Big) + \Big( b + ( d + f) \Big) i\textrm{ (From addition of complex numbers)} \\
                                               &= \Big( (a + c) + e \Big) + \Big( (b + d) + f \Big) i\textrm{  (from associativity of real numbers)} \\
                                               &= \Big( (a + c) + (b +d )i \Big) + (e + fi) \textrm{ (Bidirectionality of addition of complex numbers)} \\
                                               &= \Big( (a+ bi) + (c + di) \Big) + (e + fi) \textrm{ (Bidirectionality of addition of complex numers)} \\
                                               &= (\alpha + \beta) + \gamma
    \end{split}
\end{equation*}
$\blacksquare$

\subsubsection*{Q6}
To prove:
\begin{equation*}
  \alpha\left(\beta\gamma \right) = \left( \alpha \beta \right) \gamma \textrm{ }\forall \alpha \textrm{,} \beta \textrm{,} \gamma \in \mathbb{C}
\end{equation*}
Assuming
\begin{align*}
  \alpha &= a + bi \\
  \beta &= c + di \\
  \gamma &= e + fi
\end{align*}
It will do us well to remember the general formula of complex number
multiplication i.e
\begin{equation*}
  (u + vi)(x + yi) = (ux - vy) + (vx + uy)i
\end{equation*}  
Proof:\\
Evaluating $(\alpha\beta)\gamma$ \\
\begin{align*}
  (\alpha\beta)\gamma &= \Big( (a+bi)(c+di) \Big)(e+fi) \\
                      &= \Big((ac - bd) + (ad + cb)i\Big)(e+fi) \\
                      &= ((ac - bd)e - f(ad +cb)) + ((ac-bd)f + e(ad + cb))i \\
                      &= (ace -bde - fad - fcb) + (acf - bdf + ead + ecb)i \\
                      &= (a(ce-fd)-b(de + fc)) + (a(de+fc) + b(ce - fd))i \\
                      &= (a + bi) ((ce-fd) + (de+fc)i) \\
                      &= (a+bi)\Big( (c+di)(e + fi) \Big) \\
                      &= \alpha (\beta\gamma)
\end{align*} 
$\blacksquare$


\subsubsection*{Q7}
To prove: $\forall \alpha \in \mathbb{C} \textrm{ , } \exists \beta \in \mathbb{C}$ such that $\alpha + \beta = 0$ \\
Assumption:
\begin{enumerate}
    \item if $\alpha \in \mathbb{C}$, then $\alpha$ can be written as $(a + bi)$ where both a and b are real numbers i.e $a,b\in \mathbb{R}$
    \item $\forall \alpha \in \mathbb{R} \textrm{,} \exists (-\alpha) \textrm{ such that } \alpha + (-\alpha) = 0$ 
\end{enumerate}
Proof:\\
Let $\alpha \in \mathbb{C}$, so from our assumption it can be written in the form (a + bi) \\
let $\beta \in \mathbb{C}$ such that $\alpha + \beta = 0$, again from our assumption $\beta$ can be written in the form (x + yi) \\
\begin{equation*}
\begin{split}
    (a + bi) + (x + yi) &= 0 \\
    (a+x) + (b + y)i &= 0 \textrm{ (from addition of complex numbers)} \\
    (a + x) + (b + y)i &= 0 + 0i \textrm{ (zero of the complex number field)}
\end{split}
\end{equation*}

from the last equation above, we equate the real and imaginary parts of both the L.H.S and the R.H.S to each other, we get \\
\begin{gather*}
    a + x = 0 \\
    b + y = 0
\end{gather*}
Using assumption (2) we arrive at $x = -a$ and $y = -b$ \\
so $\beta = (-a + (-b)i) = -(a+bi) = -\alpha$ \\

Proof of uniqueness: (Proof by contradiction) \\
There exists another unique complex number $\gamma $ such that $\alpha + \gamma = 0$
\begin{align*}
    \alpha + \gamma &= 0 \textrm{ (From assumption above)} \\
    \alpha + \beta &= 0 + \beta \textrm{ (Adding $\beta$ to both sides)} \\
    0 + \gamma &= \beta \textrm{ (From proven result and concept of 0)} \\
    \gamma = \beta
\end{align*}
This disproves our assumption, hence $\beta$ is unique in that $\forall \alpha \in \mathbb{C} \textrm{ , } \exists \beta \in \mathbb{C}$ such that $\alpha + \beta = 0 \textrm{ and $\beta$ is unique}$ 
$\blacksquare$


\subsubsection*{Q8}
To prove: $\forall \alpha \in \mathbb{C} \textrm{ , with } \alpha \neq 0
\textrm{ , } \exists \beta \in \mathbb{C} \textrm{ , } (\alpha\beta =1) \land
(\beta\textrm{ is unique})
$ \\
Assumption: 
\begin{enumerate}
\item if $\alpha \in \mathbb{C}$, then $\alpha$ can be written as $(a + bi)$ where both a and b are real numbers i.e $a,b\in \mathbb{R}$
\end{enumerate}
Proof: \\
For a general complex number $\alpha$ which can represented as $a + bi$ (from
assumption) we will try to find a complex number $\beta$ such that $\alpha\beta
= 1$ whose components (real
and imaginary) can be expressed in terms $\alpha$'s real and imaginary
components (i.e a and b) \\
We take $\beta = c + di$ \\
\begin{align*}
  \alpha\beta &= 1 \\
  (a + bi)(c + di) &= 1 \\
  (ac-bd) + (ad+bc)i &= 1 + 0i \textrm{ from multiplication of two complex numbers} \\
\end{align*}
Equating the real and complex parts of the L.H.S to the R.H.S \\
\begin{gather}
  ac - bd = 1 \\
  ad + bc = 0 
\end{gather}
Multiplying (1) with -b and (2) with a we have
\begin{alignat*}{2}
    & -abc &+ b^{2}d & = -b \\
    +\hspace*{.5em} &\phantom{-}abc &+ a^{2}d & = 0 \\
    \cline{1-4} \\
    && (a^2 + b^2)d &  = -b
\end{alignat*}
and therefore we have  $d =\frac{-b}{ a^2 + b^2}$ and if we plug this into (2)
we have
\begin{align*}
  ac &= 1 + bd \textrm{ (from (2))} \\
  ac &= 1 + b(\frac{-b}{a^2 + b^2}) \\
  ac &= 1 - \frac{b^2}{a^2 + b^2} \\
  c &= \frac{a^2}{a(a^2 + b^2)} \\
  c &= \frac{a}{a^2 + b^2}
\end{align*}
Therefore $\forall \alpha \in \mathbb{C} \textrm{ , with } \alpha \neq 0
\textrm{ , } \exists \beta \in \mathbb{C} \textrm{ , } \alpha\beta =1 \textrm{
  where } \beta = \frac{a}{a^2 + b^2} + \frac{-b}{a^2 + b^2}i$ \\
\textbf{Proof of uniqueness (proof by contradiction):}\\
Let's assume there exists another complex number other than $\beta$, $\gamma$ such that
$\alpha\gamma=1$
\begin{align*}
  \alpha\gamma &= 1 \textrm{ (from assumption)} \\
  (\alpha\gamma)\beta &= \beta \textrm{ (multiplying both sides with $\beta$)} \\
  (\gamma\alpha)\beta &= \beta \textrm{ (commutativity of multiplication on $\mathbb{C}$)}\\
  \gamma(\alpha\beta) &= \beta \textrm{ (associativity of multiplication on $\mathbb{C}$)}\\
  \gamma.1 &= \beta \textrm{ (from above construction of beta such $\alpha\beta = 1$)} \\
  \gamma &= \beta  \textrm{ (multiplicative identity defined on $\mathbb{C}$)}
\end{align*}
disproving our assumption and hence $\beta$ is unique $\blacksquare$
\end{document}