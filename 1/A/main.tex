\documentclass{article}
\usepackage[utf8]{inputenc}
\usepackage{amsmath}
\usepackage{amssymb}


\begin{document}

\subsubsection*{Q1}
\begin{equation*}
\begin{split}
    \frac{1}{a + b i} &= \frac{1}{a+ bi}* \frac{a - bi}{a - bi} \\
    &= \frac{a - bi}{a^{2} + abi - abi +b^2} \\
    &= \frac{a-bi}{a^2+b^2} \\
  c + di  &= \frac{a}{a^2+b^2} - \frac{bi}{a^2 + b^2}
\end{split}    
\end{equation*}
therefore $c = \frac{a}{a^2+b^2}$ and $d = \frac{-b}{a^2+b^2}$ \\


\subsubsection*{Q2}
\begin{equation*}
    \left( \frac{-1 + \sqrt{3}i}{2} \right)^3 = \frac{1}{8}*\left(-1 + \sqrt{3}i \right)^3
\end{equation*}
using the identity $\left(a + b \right)^3 = a^3 +3a^2b + 3ab^2 + b^3$ 
where $a = -1$ and $b = \sqrt(3)i$
we have
\begin{equation*}
    \begin{split}
        \frac{1}{8} * \left(-1 + \sqrt{3}i \right)^3 &= (-1)^3 + 3*(-1)^2*(\sqrt{3}i) + 3*(-1)*(\sqrt{3}i)^2 + (\sqrt{3}i)^3 \\
        &= \frac{1}{8} * \left( -1 + 3*\sqrt{3}i + 9 -3*\sqrt{3}i \right)  \\
        &= \frac{8}{8} \\
        &= 1
    \end{split}
\end{equation*}

\subsubsection*{Q3}
$-\sqrt{i}$ and $+\sqrt{i}$ are two distinct square roots of $i$ \\ 

\subsubsection*{Q4}
To show: 

\begin{equation*}
    \alpha + \beta = \beta + \alpha \textrm{ }\forall \alpha \textrm{,}\beta \in \mathbb{C}
\end{equation*}
let $\alpha = a +bi$ and $beta = c + di$ \\
by definition of addition of complex numbers \\
\begin{equation}
(a + bi) + ( c + di) = (a+c) + ( b + d)i
\end{equation}
by definition of complex numbers we know a,b,c,d $\in \mathbb{R}$ \\
so addition is commutative in $\mathbb{R}$ \\
as a consequence $ a + b = b + a$ and $ c + d = d + c $ \\
there $(1)$ becomes
\begin{equation*}
\begin{split}
    (a + bi) + (c + di) &= (a + c) + (b + d)i \\
     &= (c + a) + (d + b)i \\
     &= (c + di) + (a +  bi)  \textrm{ (from bidirectionality of how complex number addition is defined) } \\
     &= \beta + \alpha 
\end{split}
\end{equation*}
$\blacksquare$

\subsubsection*{Q5}
To prove:
\begin{equation*}
    \alpha + \left( \beta + \gamma \right) = \left( \alpha + \beta \right) + \gamma \textrm{ }\forall \alpha \textrm{,} \beta \textrm{,} \gamma \in \mathbb{C}
\end{equation*}
Assuming $\alpha = a + bi$, $\beta = c + di$, and $\gamma = e + fi$ \\
evaluating $\alpha + \left( \beta + \gamma \right)$
\begin{equation*}
    \begin{split}
        \alpha + \left( \beta + \gamma \right) &= (a + bi) + \Big( (c + di ) + (e + fi) \Big) \\
                                               &= (a + bi) + \Big( (c + e) + (d + f)i \Big)  \textrm{ (From addition of complex numbers)}\\
                                               &=  \Big(a + ( c + e) \Big) + \Big( b + ( d + f) \Big) i\textrm{ (From addition of complex numbers)} \\
                                               &= \Big( (a + c) + e \Big) + \Big( (b + d) + f \Big) i\textrm{  (from associativity of real numbers)} \\
                                               &= \Big( (a + c) + (b +d )i \Big) + (e + fi) \textrm{ (Bidirectionality of addition of complex numbers)} \\
                                               &= \Big( (a+ bi) + (c + di) \Big) + (e + fi) \textrm{ (Bidirectionality of addition of complex numers)} \\
                                               &= (\alpha + \beta) + \gamma
    \end{split}
\end{equation*}
$\blacksquare$

\subsubsection*{Q6}




\end{document}
